% !TEX TS-program = pdflatex
% !TEX encoding = UTF-8 Unicode
 
\documentclass[12pt]{report} % use larger type; default would be 10pt
 
\usepackage[utf8]{inputenc} % set input encoding (not needed with XeLaTeX)
 
%%% Examples of Article customizations
% These packages are optional, depending whether you want the features they provide.
% See the LaTeX Companion or other references for full information.
 
%%% PAGE DIMENSIONS
\usepackage{geometry} % to change the page dimensions
\geometry{a4paper} % or letterpaper (US) or a5paper or....
\geometry{margin=1in} % for example, change the margins to 2 inches all round
\geometry{landscape} % set up the page for landscape
%read geometry.pdf for detailed page layout information
 
\usepackage{graphicx} % support the \includegraphics command and options
 
% \usepackage[parfill]{parskip} % Activate to begin paragraphs with an empty line rather than an indent
 
%%% PACKAGES
\usepackage{booktabs} % for much better looking tables
\usepackage{array} % for better arrays (eg matrices) in maths
\usepackage{paralist} % very flexible & customisable lists (eg. enumerate/itemize, etc.)
\usepackage{verbatim} % adds environment for commenting out blocks of text & for better verbatim
\usepackage{subfig} % make it possible to include more than one captioned figure/table in a single float
\usepackage{amsmath}
% These packages are all incorporated in the memoir class to one degree or another...
 \graphicspath{{Figures/}}
%%% HEADERS & FOOTERS
\usepackage{fancyhdr} % This should be set AFTER setting up the page geometry
\pagestyle{fancy} % options: empty , plain , fancy
\renewcommand{\headrulewidth}{0pt} % customise the layout...
\lfoot{}\cfoot{\thepage}
 
\rfoot{}
 
%%% SECTION TITLE APPEARANCE
\usepackage{sectsty}
\allsectionsfont{\sffamily\mdseries\upshape} % (See the fntguide.pdf for font help)
% (This matches ConTeXt defaults)
 
 
%%% END Article customizations
 
%%% The "real" document content comes below...
 
\title{\bf Flying Carpet Testing and ROTCFD Analysis\\  }
\author{\bf Micaiah Smith-Pierce
\\ Experimental Aerodynamics and Concepts Group
\\Daniel Guggenheim School of Aerospace Engineering
\\Georgia Institute of Technology
\\Atlanta GA 30332-0150
}
\date{\it Updated January 23, 2017} % Activate to display a given date or no date (if empty),
 
         % otherwise the current date is printed 

\usepackage{graphicx}
 
\begin{document}
\maketitle
 
\tableofcontents
 
\chapter{Abstract}

The author was introduced to the EACG, its projects, and standard practices.  He acquired the proper safety training and
was assigned to work on Glitter Belt (particularly the Flying Carpet concept) and ROTCFD projects.  He was made familiar
with the nature of CFD while awaiting of licence renewal for the necessary software.  He was informed of the prior work on
the Flying Carpet and the need for demonstration and testing models.  A Flying Carpet wind tunnel model was designed and
constructed, and is ready for testing.  Concepts for a more advanced model with a deployable sheet were considered.

\chapter{Introduction}

The author's work in the EACG is divided among two projects: Glitter Belt and ROTCFD.  The former is concerned with using high
altitude aircraft, carrying reflectors, to stop or reverse global warming, and the latter is concerned with using computational
techniques to model the flow around
the EACG's various test models to suppliment experimental results.  The two projects and the author's work on them are introduced
separately below.

\subsection{Glitter Belt}

The Glitter belt project aims to reverse climate change by reflecting solar radiation out to space.  The reflection will be accomplished
by solar-powered aicraft carrying ultralight mylar sheets at approximately 100,000 feet of altitude.  Cost analysis shows that this is
feasible to do using government funding.  The name "Glitter Belt" refers to the appearance the reflectors may have when viewed from space.

There are three different concepts for implementing the Glitter Belt: The Flying Carpet, the Quadrotor, and the Baloon Beanie.  The
author's work concerns primarily the first, which includes more challenging aerdynamic questions,
appropros of this lab group's title and purpose.  In the Flying Carpet,
the first, the mylar sheet is supported by aerodynamic lift.  During the day, it is towed through the air by propellers, driven
by electric motors.  The propellers are mounted on a flying wing, and the motors are powered by solar cells on the wing.  During the
night, the aircraft maintains forward flight by gliding downward, using gravitatational potential energy, staying above the upper limit
of controlled airspace, 60,000 feet.  Incidentally, this concept may also be useful for transportation on Mars, since the martian atmosphere
at lower altitude is similar to that of earth at 100,000 feet.

The second concept is the Quadrotor.  This involves supporting the sheets using four rotary wings.  Thus far no feasible way has been
found to keep such an aircraft above 60,000 feet at night, so the author's work is not concerned with it.

The third and final concept is the Balloon Beanie.  In this concept, a flat reflector sheet is supported by hydrogen balloons.  Some
solar powered rotors are included to provide trim and propulsion. (It will be necessary under certain conditions to move the aircraft,
although most of the time it will drift in the wind.)  This may be particularly useful near the poles, where due to the low angle of
elevation of the sun, a horizontal reflector such as the Flying Carpet will be less effective.
and the later nearer to the equator.

\subsection{ROTCFD}

CFD (Computational Fluid Dynamics) is a branch of aerdynamic research where computational methods are used to solve the Navier-Stokes
equations in order to model the flow around a body.  It is necessary to model the flow around our test models in addition to obtaining
experimental results, in order to better understand the flow around them.

\chapter{Define Objectives}
 

{\bf Objectives:}
\begin{itemize}
\item Creat CFD simulations of wind tunnel experiments to the desired degree of accuracy
\item Build a scale model of the Flying carpet which can:
  \begin{itemize}
  \item carry a reflector sheet internally, as in the climb phase of the mission
  \item deploy the sheet
  \item hold the sheet steady (without significant flutter) during a wind tunnel test
  \end{itemize}
\end{itemize}

\chapter{Prior Work}

Cost analysis shows that the Glitter Belt project can be implimented using government expenditures.  It also shows that the flying carpet
will probably be cheaper to produce per unit reflector area than the Balloon Beanie.  However, the Balloon Beanie has the unique property
of being capable of orienting to be normal to the Sun's rays no matter the angle.  This eliminates the need to place them on the part of
the Earth directly beneath the Sun, near the poles in particular (for the purpose of stopping ice melting).  The author suggests that
both concepts may be manufactured, and the Flying Carpet may be deployed beneath the Sun and the Balloon Beanie may be deployed near the
poles.

The primary challenge of the Flying Carpet concept is keeping the reflector sheet smooth and flat.  Present design calls for a sheet
of reflective mylar trailing behind the wing.  However, wind tunnel testing has shown that the sheet oscillates in a self-excited manner.
 When the aspect ratio is high, the oscillations propagate spanwise.  When the aspect ratio is low (i.e. less than 1) the oscillations
are longitudinal.  Moving the sheet away from the wing, or making spanwise slits in it does not help.  Limited success has been achieved
by introducing rigid structures made of drinking straws into the sheet.  The oscillations are detrimental in that they increase drag
and reduce the effective area of the sheet.

Another design calls for stretching the sheet by its four corners which will be connected to a rigid frame.  This causes a problem because
the sheet bends upward like a parachute, which has inferior aerodynamic and reflective characteristics.

\chapter{Project Schedule}
\begin{itemize}
  \item Monday morning 2-5-17:  Show existing model to Prof. Komerath and obtain directives for further development/testing
  \item Monday afternoon 2-5-17:
  \begin{itemize}
    \item Make any necessary changes to the model
    \item Finalize deployable sheet concept and design model
    \item Check availability of CFD software
  \end{itemize}
  \item Friday 2-9-17:
  \begin{itemize}
    \item Begin construction of deployable sheet model
    \item Begin CFD work if software is available
  \end{itemize}
\end{itemize}

\chapter{Experimental Setup}
An experiment has yet to be formally designed.  It will involve attempting to float the model using the wind generated by the wind tunnel's
ventilation fan, and running wind tunnel tests on the model to see which configurations minimize sheet oscillations.

\subsection{Model Details}

The model will be made primarily of styrofoam and balsa wood.  The reflective sheet will be 60cm span by 30cm chord.  It will be attatched
to the top of the winglets using a dowel 92cm long.  The wing will be of rectangular planform, 71cm span by 10cm chord.  The winglets will
be 10.5 cm high by 10cm chord.  They will connect to the wing by means of two pins each.  By using pins bent at different angles and
inserting them into different holes in the winglets, the winglets may be installed at any di/anhedral and sweep angle.
The wings will be constructed primarily of styrofoam.  A balsa wood spar may be inserted for additional stiffness and robustness to bending
moments.

Before designing this model, the author and his colleagues were given rather various mission requirements.  The need to make a model that
could support its own weight using lift was posited, in addition to the need for a modular test model where the different winglet configurations
could be tested.  This model is expected to meet both requirements by being both modular and light.

\subsection{Load Cell Details}
A load cell has not been selected.

\subsection{Calculations of expected forces and moments}
The priority thus far has been to produce a model, not to conduct analysis.

\subsection{Static testing of the model}
Testing procedures are TBD.

\chapter{Flow and Test Conditions}

The model will be tethered and exposed to gentle wind in the wind tunnel doorway to test its ability to support itself.
 Then the model will be held at fixed low angle of attack to observe sheet osciliations.

\chapter{Expected results and plots}
Not yet determined


\chapter{Conclusions}

Now that information concerining objectives and previous work has been obtained, model designs are being finalized.  The Flying Carpet model
can be expected to be complete soon (next week at the latest).  CFD work will commence as soon as software is available.

\begin{thebibliography}{9}
\end{thebibliography}


\end{document}
